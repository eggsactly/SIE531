\documentclass[12pt]{article}
\usepackage{lingmacros}
\usepackage{tree-dvips}
\usepackage{amsmath}
\usepackage{accents}
\newcommand{\ubar}[1]{\underaccent{\bar}{#1}}
\usepackage{hyperref}
\hypersetup{
    colorlinks=true,
    linkcolor=blue,
    filecolor=magenta,      
    urlcolor=cyan,
}
\begin{document}

\title{SIE 531 Comprehensive Notes}
\date{April 17, 2018}
\maketitle

\section*{Definitions}

\begin{itemize}

\item Model: A method to study and analyze a system.

\item Physical Model: Physical replica or scale model of the system. 

\item Analytical Model: A mathmatical model that represents a quantity in the system as an equation.

\item Simulation Model: Highest fidelity model, used when the system characteristics are \emph{complex, dynamic, uncertain}.

\item Static: Where time plays no role.

\item Dynamic: Where time plays a role. 

\item Continuous: Where the state of the system can change instantaniously over time.

\item Discrete: Where the state of the system can only change at separated points in time.

\item Deterministic: Where the state of the system can be found from knowing the initial conditions of the system.

\item Stochastic: Where the state of the system is uncertain. 

\item Validation: Does the model follow reality?

\item Verification: Does the model work the way it was intended? 

\item Sample Path: A record (an instance realization) of TD behavior of the system.

\item Entity: Dynamic objects in the system, such as a customer. 

\item Attribute: A variable characterizing an entity. 

\item Variable: Information that reflects some characteristic of the system. 

\item Resource: A type of object that can be seized and released by an object to perform a task.

\item Queue: An area for an Entity to wait while a Resource become available

\item Statistical Accumulators: Variables needed for storing statistical information necessary to estimate desired performance measures. 

\item Event: Something that happens at an instant of simulated time that may change a system state. 

\item Initialization Event: Sets things up in the system.

\item Arrival Event: When a new Entity enters the system.

\item Departure: When an Entity leaves the system.

\item End: When the simulation stops.

\item Event Calendar: Where future events are stored for the system. 

\item Simulation Clock: The current value of simulation clock TNOW.

\item Scheduling: Creating a future event and placing it in the event calendar. 

\end{itemize}

\section*{Flow Control of Simulation}

\begin{enumerate}
\item Initialize
\begin{itemize}
\item Initialize all data structs
\item Schedule the first events
\end{itemize}
\item Main (Feeds back and forth to timing routine)
\begin{enumerate}
\item Initialize
\item Timing Routine
\item Event Routine
\end{enumerate}
\item Timing Routine
\begin{enumerate}
\item Grab event
\item Update T-NOW with even time of the grabed event
\item send the event information
\end{enumerate}
\item Event Routine (Goes to end condition satisfied, which goes back to main)
\begin{enumerate}
\item ? Watch lecture 1/26/18 at 30 mins in to get this info
\item Update statistical counter 
\item ?
\end{enumerate}
\end{enumerate}




\end{document}
